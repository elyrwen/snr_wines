\documentclass[10pt,a4paper]{article}
\usepackage[utf8x]{inputenc}
\usepackage[polish, english]{babel}
\usepackage[T1]{fontenc}
\usepackage{ucs}
\usepackage{amsmath}
\usepackage{amsfonts}

%\usepackage{amssymb} %rzuca błędem

\author{Paweł Z., Paweł S.}
\title{Klasyfikator jakości Win portugalskich wykorzystujący SVM}
\begin{document}
\maketitle
\section{Instalacja}
Używamy wersji Shogun 0.9.3
\subsection{Linux}

Na platformę Linux, Shogun dostępny jest w wersji binarnej dla dystrybucji Debian i Ubuntu.
W przypadku pozostałych dystrybucji konieczna jest kompilacja źródeł, która wymaga, w zależności od 
instalowanych modułów, dodatkowych bibliotek. Proces kompilacji przebiega standardowo przy użyciu skryptu configure oraz make'a, jednak nie jest on pozbawiony błędów, o których nie wspomniano w dokumentacji.

Na przykład dla Fedory 13 instalacja interfejsu Shogun Octave wymaga zainstalowania pakietów octave-devel, swig. Pakiety można automatyczne zainstalować używając menedżera pakietów yum. W fazie konfiguracji kompilacji, do polecenia ./configure należy dołączyć flagę --disable-hdf5, ze względu na niekompatybilność pakietu Shogun z biblioteką hdf5 na Fedorze 13. Niedołączenie flagi powoduje błąd kompilacji. 


\subsection{Windows}

Według oficjalnej strony projektu pakiet Shogun możne być wykorzystywany pod systemami Windows poprzez kompilację z wykorzystaniem oprogramowania Cygwin.
Niestety nie jest to takie proste i nie wszystkie elementy pakietu udają się uruchomić. Przykładem może być moduł przeznaczony dla Matlaba, który praktycznie nie daje
się uruchomić. Informację o tym można znaleźć jednym z forów Shoguna.


\end{document}
